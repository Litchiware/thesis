\chapter{总结与展望}

\section{论文总结}

随着科技和社会的进步,以及人们对工业生产过程自动化程度的需求提升,
现在很多的工业系统开始朝着大规模集成的方向发展,而且单个设备的结
构也越来越复杂,系统各组成部分之间相互耦合和依赖的程度也逐渐加强。
这样的发展趋势一方面为工业设备实现更加复杂、精细以及智能的任务奠
定了基础,另一方面,工业系统的高度集成化、设备的高度复杂化以及各
环节之间紧密的耦合程度带来的问题就是系统发生故障的可能性增大、故
障发生所带来的后果的严重性增强。因此,研究和发展故障诊断技术对于
提升复杂工业系统的可靠性至关重要。

本课题主要研究了基于数据的故障诊断算法。首先在信号片段的RDFT频谱
序列的基础上,分别设计了两种信号单一尺度的特征提取方法,并在此基
础上提出了一种基于深度神经网络的多尺度特征提取方法。本文中设计的
每一种故障诊断模型都在CWRU滚动轴承故障数据库上进行了仿真验证。

本文首先对目前常见的故障诊断算法进行了综述。整体上来看,故障诊断
算法主要分为两大类,第一类是基于模型的算法,这一类算法提出的相对
较早,发展的也比较成熟,目前应用的也更加广泛,缺点是算法非常依赖
于对系统建模的精确程度,在很多大型复杂系统中很难完成精确建模。第
二类是基于数据的算法,这一类算法随着近些年人工智能的快速发展越来
越受到重视,其优势在于不用关心系统内在的运行机理,通过分析系统在
长期的运行过程中产生的各种历史数据就可以学习系统的故障模式,从而
对系统的健康状态做出推断;其劣势在于算法的效果依赖于历史数据,在
缺少数据或者数据量不够的情况中很难得到应用,此外很多人工智能算法
的训练过程需要非常大的计算量,因此对故障诊断系统的在线学习速度形
成了制约,不过这在以后将会逐渐得到改善。

本文接下来设计了两个基于单一尺度信号特征提取方法的故障诊断模型,
以及一个基于多尺度信号特征提取的故障诊断模型,并且对比了各自在
滚动轴承故障数据库CWRU上的诊断效果,验证了多尺度特征提取过程在
故障诊断算法中的优势。简要总结如下:

1)基于单层下采样的特征提取方法和基于SVM的级联分类模型。这一部分
对RDFT变换以及下采样操作在特征提取中的效果进行了验证。本文中的三
个故障诊断模型都是以信号的RDFT频谱序列作为输入而不是原始信号片段,
以此来避免各信号片段的随机截取过程造成的相位差异所带来的影响。这
部分针对数据样本的类别间固有的层次性设计了一个级联分类过程,并且
使用单层下采样在RDFT频谱序列上提取的特征完成了模型的训练和预测过
程。在仿真实验中,我们还考察了下采样的核宽度对模型最终的预测精度
的影响,最后给出在测试集上最好的预测精度是93.81\%。

2)基于卷积神经网络设计了一个类似VGG结构的模型来完成对信号频谱序
列的特征提取和分类。虽然这个模型提取的也是单一尺度下的特征,但是
与前一个模型相比,在特征提取阶段由于卷积层和多个下采样层的存在,
我们可以在每个下采样层使用最小的核宽度2,避免过多信息的丢失,从而
完成逐层的特征提取过程。在仿真实验中给出该模型在测试集上的预测精
度是98.27\%。

3)在第二种模型的基础上,向网络中加入了一种新的操作——上采样层。然
后通过一个递归过程,建立了一个具有多尺度特征提取能力的网络结构。
并且基于这个多尺度网络结构实现特征提取过程以及故障分类过程。在这
个多尺度网络结构中,我们在每层递归中都将网络分为两个分支,并且对一
个分支中的特征进行下采样和上采样,最后将两个分支融合的结果作为输出。
这样设计的优势在于能够将下采样前包含更多细节信息的结果保留下来,并
且跟下采样后获得的包含更多整体特征的结果进行融合,提升模型对输入信
号的理解能力。在仿真实验中给出该模型在测试集上的预测精度是99.93\%。

通过本文中三个模型的比较以及各自仿真实验给出的结果可以看出,多尺度
特征提取对模型理解输入信号的程度以及模型的诊断精度的提升是很明显的。
这对我们在故障诊断算法的研究过程中提供了一个很重要的思路:有效地融
合信号的细节特征和信号的整体特征。

\section{研究展望}

由于故障诊断模型应用对象之广,输入数据形式的变化之多,以及近些年来
人工智能的飞速发展,本文只是对基于数据的故障算法进行了很小一部分的
研究,对基于多尺度特征提取的方法进行了初步的探索,未来还有很多需要
深入展开研究的内容,主要列举如下:

1)研究其它形式的神经网络(例如RNN、R-CNN等)在对信号的特征提取及
在故障诊断中的应用。

2)针对多尺度特征提取方法,调研其他领域的类似方法的研究进展,对这
一类方法在故障诊断中的应用进行更加深入的研究。

3)结合相关实际工业系统,将本文提出的故障诊断模型有效地运用到这些
实际系统当中。
