\chapter{总结与展望}
\label{cha:summary}

\section{论文总结}

随着科技和社会的进步,以及人们对工业生产过程自动化程度的需求提升,
现在很多的工业系统开始朝着大规模集成的方向发展,而且单个设备的结
构也越来越复杂,系统各组成部分之间相互耦合和依赖的程度也逐渐加强。
这样的发展趋势一方面为工业设备实现更加复杂、精细以及智能的任务奠
定了基础,另一方面,工业系统的高度集成化、设备的高度复杂化以及各
环节之间紧密的耦合程度带来的问题就是系统发生故障的可能性增大,故
障发生所带来的后果的严重性增强。因此,研究和发展故障诊断技术对于
提升复杂工业系统的可靠性至关重要。

传统的基于模型的算法过于依赖对系统的精确建模,这在很多大型复杂系
统中很难满足。数据驱动的故障诊断算法随着近些年人工智能的快速发展
越来越受到重视,其优势在于不用关心系统内在的运行机理,通过分析系
统在长期的运行过程中产生的各种历史数据就可以学习系统的故障模式,
从而对系统的健康状态做出推断。如何从大量过程数据中提取潜在故障特
征是决定数据驱动的故障诊断方法效果好坏最关键的因素。实际物理系统
由于多种组成部件以及所处环境间的相互作用,通常带有多种时间尺度的
特征,基于单一尺度特征的故障诊断方法难以有效挖掘和利用过程数据中
包含的多种尺度的信息。深度神经网络具有强大的数据拟合能力以及对数
据深层特征的挖掘能力,因此非常适合用于多尺度故障诊断方法中进行多
种尺度特征的提取和故障模式的识别,因此本文主要研究了基于深度神经
网络的多尺度故障诊断方法,研究内容简要总结如下:

1)首先基于下采样和级联SVM设计了一种单一尺度的故障诊断方法,对下
采样操作用于信号特征提取的效果进行了验证。将信号的RDFT频谱序列作
为模型的输入可以避免信号片段的随机截取过程造成的相位差异所带来的
影响。针对数据样本的类别间固有的层次性设计了一个级联分类过程,并
且使用单层下采样在RDFT频谱序列上提取的特征完成了模型的训练和预测
过程。实验中考察了下采样的核宽度对模型最终预测精度的影响,证明下
采样核宽度过大或过小都会降低诊断精度,最后得到在测试集上最高的预
测精度是91.73\%。

2)接下来基于深度神经网络设计了一种单一尺度的故障诊断方法,实现了信号特
征的逐层提取,验证了深度神经网络用于信号特征提取和故障模式分类的
强大能力。网络采用了类似VGG的结构,虽然这个模型提取的也是单一尺度
下的特征,但是与前一个模型相比,在特征提取阶段由于卷积层和多个下采
样层的存在,我们可以在每个下采样层使用最小的核宽度2,避免过多信息
的丢失,从而完成逐层的特征提取过程。实验得出该模型在测试集上的预
测精度是97.42\%。

3)最后提出了一种基于深度神经网络的多尺度故障诊断方法,实现了多种尺度
特征的自动提取与融合。模型中新加入了上采样层,通过一种递归的方式
建立了一个具有多尺度特征提取和融合能力的网络结构。每层递归中都将
网络分为两个分支,其中一个分支上的特征进行下采样、卷积和上采样操
作,另一个分支上只进行卷积操作,最后将两个分支相加的结果作为输出
返回上层递归。这样设计的优势在于能够将下采样前包含更多细节信息的
结果保留下来,并且跟下采样后获得的包含更多整体特征的结果进行融合,
从而提升模型对输入信号的理解能力。在仿真实验中给出该模型在测试集
上的预测精度是99.85\%。

通过比较实验结果可以看出,多尺度特征提取方法对于故障诊断模型从过
程数据中提取深层故障信息从而提升模型的诊断精度是至关重要的。

\section{研究展望}

由于故障诊断模型应用对象之广,输入数据形式的变化之多,以及近些年来
人工智能的飞速发展,本文只是对基于数据的故障算法进行了很小一部分的
研究,对基于多尺度特征提取的方法进行了初步的探索,未来还有很多需要
深入展开研究的内容,主要列举如下:

1)研究其它形式的神经网络(例如RNN、R-CNN等)在对信号的特征提取及
在故障诊断中的应用。

2)针对多尺度特征提取方法,调研其他领域的类似方法的研究进展,对这
一类方法在故障诊断中的应用进行更加深入的研究。

3)结合相关实际工业系统,将本文提出的故障诊断模型有效地运用到这些
实际系统当中。
