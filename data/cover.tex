\thusetup{
  %******************************
  % 注意:
  %   1. 配置里面不要出现空行
  %   2. 不需要的配置信息可以删除
  %******************************
  %
  %=====
  % 秘级
  %=====
  %secretlevel={秘密},
  %secretyear={10},
  %
  %=========
  % 中文信息
  %=========
  ctitle={基于深度神经网络的多尺度故障诊断方法研究},
  cdegree={工学硕士},
  cdepartment={自动化系},
  cmajor={控制科学与工程},
  cauthor={李亮民},
  csupervisor={李志恒副教授},
  %cassosupervisor={陈文光教授}, % 副指导老师
  %ccosupervisor={某某某教授}, % 联合指导老师
  % 日期自动使用当前时间,若需指定按如下方式修改:
  % cdate={超新星纪元},
  %
  % 博士后专有部分
  %cfirstdiscipline={计算机科学与技术},
  %cseconddiscipline={系统结构},
  %postdoctordate={2009年7月——2011年7月},
  %id={编号}, % 可以留空: id={},
  %udc={UDC}, % 可以留空
  %catalognumber={分类号}, % 可以留空
  %
  %=========
  % 英文信息
  %=========
  etitle={Research on Multi-scale Fault Diagnosis Method Based on Deep Neural Network},
  % 这块比较复杂,需要分情况讨论:
  % 1. 学术型硕士
  %    edegree:必须为Master of Arts或Master of Science(注意大小写)
  %             “哲学、文学、历史学、法学、教育学、艺术学门类,公共管理学科
  %              填写Master of Arts,其它填写Master of Science”
  %    emajor:“获得一级学科授权的学科填写一级学科名称,其它填写二级学科名称”
  % 2. 专业型硕士
  %    edegree:“填写专业学位英文名称全称”
  %    emajor:“工程硕士填写工程领域,其它专业学位不填写此项”
  % 3. 学术型博士
  %    edegree:Doctor of Philosophy(注意大小写)
  %    emajor:“获得一级学科授权的学科填写一级学科名称,其它填写二级学科名称”
  % 4. 专业型博士
  %    edegree:“填写专业学位英文名称全称”
  %    emajor:不填写此项
  edegree={Master of Science},
  emajor={Control Science and Engineering},
  eauthor={Li Liangmin},
  esupervisor={Associate Professor Li Zhiheng},
  %eassosupervisor={Chen Wenguang},
  % 日期自动生成,若需指定按如下方式修改:
  % edate={December, 2005}
  %
  % 关键词用“英文逗号”分割
  ckeywords={故障诊断, 多尺度, 特征提取, 支持向量机, 深度神经网络},
  ekeywords={Fault Diagnosis, Multi-scale, Feature Extraction, SVM, Deep Neural Network}
}

% 定义中英文摘要和关键字
\begin{cabstract}
  随着工业生产过程自动化程度的日益提升,越来越多的工业系统朝着大规
  模集成的方向发展,单个设备的结构也越来越复杂,而故障诊断技术正是
  保障这些大型复杂工业系统可靠运行的关键所在。
  
  如何从大量过程数据中提取潜在故障特征是决定数据驱动的故障诊断方法
  效果好坏最关键的因素。实际物理系统由于多种组成部件以及所处环境间
  的相互作用,通常带有多种时间尺度的特征,基于单一尺度特征的故障诊
  断方法难以有效挖掘和利用过程数据中包含的多种尺度的信息。深度神经
  网络具有强大的数据拟合能力以及对数据深层特征的挖掘能力,因此非常
  适合用于多尺度故障诊断方法中进行多种尺度特征的提取和故障模式的识
  别。本文研究了基于深度神经网络的多尺度故障诊断方法,主要工作包括:
  
  1)基于下采样和级联SVM设计了一种单一尺度的故障诊断方法,对下采样
  操作用于信号特征提取的效果进行了验证;实验证明下采样核宽度过大或
  过小都会降低诊断精度,这是采用单层下采样进行特征提取的局限。
  
  2)基于深度神经网络设计了一种单一尺度的故障诊断方法,实现了信号特
  征的逐层提取,验证了深度神经网络用于信号特征提取和故障模式分类的
  强大能力;实验证明多个卷积层和下采样层堆叠的结构在提取主要特征的同
  时能够避免过多信息的丢失。
  
  3)提出了一种基于深度神经网络的多尺度故障诊断方法,实现了多种尺度
  特征的自动提取与融合。模型中新加入了上采样层,通过一种递归的方式
  建立了一个具有多尺度特征提取和融合能力的网络结构。每层递归中都将
  网络分为两个分支,其中一个分支上的特征进行下采样、卷积和上采样操
  作,另一个分支上只进行卷积操作,最后将两个分支相加的结果作为输出
  返回上层递归。这样设计的优势在于能够将下采样前包含更多细节信息的
  结果保留下来,并且跟下采样后获得的包含更多整体特征的结果进行融合,
  从而提升模型对输入信号的理解能力。

  实验证明,本文提出的基于深度神经网络的多尺度故障诊断方法比其他常
  见的多尺度故障诊断方法的诊断精度更高。

\end{cabstract}

% 如果习惯关键字跟在摘要文字后面,可以用直接命令来设置,如下:
% \ckeywords{\TeX, \LaTeX, CJK, 模板, 论文}

\begin{eabstract}
  With the manufacturing automation getting increasingly popular,
  Industrial systems have become large-scale and highly integrated,
  and the structure of a single device is also very complex. Hence,
  fault diagnosis technology is the key to reliable operation of
  these large and complex systems.

  The way to extract potential fault characteristics from a large
  number of process data is critical to the effect of data-driven
  fault diagnosis methods. Single-scale feature based fault diagnosis
  methods can not effectively exploit and utilize the information
  of various scales in the process data. The deep neural network
  has powerful data fitting capability and can dig deep features,
  so it is very suitable to be used in designing multi-scale fault
  diagnosis method to extract and classify fault features. In this
  dissertation, the multi-scale fault diagnosis methods based on deep
  neural network is studied. The main work is as follows:

  1) Based on the downsampling operation and a cascaded SVM classifier,
  a single-scale fault diagnosis method is designed to verify the
  effectiveness of the downsampling operation in signal feature extraction.
  Experiments show that whether the kernel width of downsampling operator
  is too large or too small, it will decrease the model's accuracy,
  which is the limitation of using one single downsampling operator
  for feature extraction. 

  2) Based on the deep neural network, a single-scale fault diagnosis
  method is designed to accomplish the layer-by-layer extraction of signal
  characteristics, which verifies the ability of deep neural network
  for signal feature extraction and fault pattern classification. Experiments
  show that the structure of multiple stacked convolutional and subsampling
  layers can avoid the loss of too much information while extracting main features.

  3) A multi-scale fault diagnosis method based on deep neural network is
  proposed, which extracting and fusing features of various scales automatically.
  In the model, upsampling layers are introduced and a network structure
  with the capability of extracting and fusing multi-scale features is
  established in a recursive way. The input of this network is processed
  in two branches recursively, one of which is downsampled, convoluted
  and upsampled and the other branch is convolved only. Then, the two branches
  are added together and returned to the upper recursion. The advantage of
  this structure is that it can preserve the results without sampling operations
  which have more detail information and fuse it with the results with
  sampling operations which have more overall features, making the model
  understand the input signal more deeply.

  Experiments show that the multi-scale fault diagnosis method based on deep
  neural network proposed in this dissertation can achieve higher accuracy
  than other common multi-scale fault diagnosis methods.
\end{eabstract}

% \ekeywords{\TeX, \LaTeX, CJK, template, thesis}
