\thusetup{
  %******************************
  % 注意:
  %   1. 配置里面不要出现空行
  %   2. 不需要的配置信息可以删除
  %******************************
  %
  %=====
  % 秘级
  %=====
  %secretlevel={秘密},
  %secretyear={10},
  %
  %=========
  % 中文信息
  %=========
  ctitle={基于深度神经网络的多尺度故障诊断方法研究},
  cdegree={工学硕士},
  cdepartment={自动化系},
  cmajor={控制科学与工程},
  cauthor={李亮民},
  csupervisor={李志恒副教授},
  %cassosupervisor={陈文光教授}, % 副指导老师
  %ccosupervisor={某某某教授}, % 联合指导老师
  % 日期自动使用当前时间,若需指定按如下方式修改:
  % cdate={超新星纪元},
  %
  % 博士后专有部分
  %cfirstdiscipline={计算机科学与技术},
  %cseconddiscipline={系统结构},
  %postdoctordate={2009年7月——2011年7月},
  %id={编号}, % 可以留空: id={},
  %udc={UDC}, % 可以留空
  %catalognumber={分类号}, % 可以留空
  %
  %=========
  % 英文信息
  %=========
  etitle={Research on Multi-scale Fault Diagnosis Method Based on Deep Neural Network},
  % 这块比较复杂,需要分情况讨论:
  % 1. 学术型硕士
  %    edegree:必须为Master of Arts或Master of Science(注意大小写)
  %             “哲学、文学、历史学、法学、教育学、艺术学门类,公共管理学科
  %              填写Master of Arts,其它填写Master of Science”
  %    emajor:“获得一级学科授权的学科填写一级学科名称,其它填写二级学科名称”
  % 2. 专业型硕士
  %    edegree:“填写专业学位英文名称全称”
  %    emajor:“工程硕士填写工程领域,其它专业学位不填写此项”
  % 3. 学术型博士
  %    edegree:Doctor of Philosophy(注意大小写)
  %    emajor:“获得一级学科授权的学科填写一级学科名称,其它填写二级学科名称”
  % 4. 专业型博士
  %    edegree:“填写专业学位英文名称全称”
  %    emajor:不填写此项
  edegree={Master of Science},
  emajor={Control Science and Engineering},
  eauthor={Li Liangmin},
  esupervisor={Associate Professor Li Zhiheng},
  %eassosupervisor={Chen Wenguang},
  % 日期自动生成,若需指定按如下方式修改:
  % edate={December, 2005}
  %
  % 关键词用“英文逗号”分割
  ckeywords={故障诊断, 多尺度, 特征提取, 支持向量机, 神经网络},
  ekeywords={Fault Diagnosis, Machine Learning, Neural Network, Multi-scale, Feature Extraction}
}

% 定义中英文摘要和关键字
\begin{cabstract}
  随着工业生产过程自动化程度的日益提升,越来越多的工业系统朝着大规
  模集成的方向发展,单个设备的结构也越来越复杂,而故障诊断技术正是
  保障这些大型复杂工业系统可靠运行的关键所在。

  本文主要研究了基于数据的故障诊断算法。我们在信号片段的RDFT频谱
  序列的基础上,分别设计了两个基于单一尺度信号特征提取方法的故障
  诊断模型,以及一个基于多尺度信号特征提取的故障诊断模型

  首先我们使用单层下采样操作在信号的RDFT频谱序列上完成了单一尺度的
  特征提取过程。并针对数据样本的类别间固有的层次性设计了一个级联分
  类器来实现特征分类。在仿真实验中,我们还考察了下采样的核宽度对模
  型最终的预测精度的影响,实验结果证明下采样核宽度过大或过小都会降
  低模型的精度,这是采用单层下采样进行特征提取带来的局限,因此我们
  考虑设计一种包含多次下采样操作的模型来完成特征提取。

  卷积神经网络利用多个卷积层和下采样层堆叠的结构,完成对输入数据逐
  层的特征提取和抽象过程。因此本文设计了一个类似VGG结构的模型来完成
  对信号频谱序列的特征提取和分类。虽然这个模型提取的也是单一尺度下
  的特征,但是与前一个模型相比,在特征提取阶段由于卷积层和多个下采
  样层的存在,我们既能够从输入的频谱序列中提取出主要特征,同时也能
  够避免过多信息的丢失。

  为了进一步提升模型的诊断精度,在前面基于CNN的模型的基础上,向网络
  中加入了一种新的操作层——上采样层。然后通过一个递归过程,建立了一
  个具有多尺度特征提取能力的网络结构。最后基于这个多尺度网络结构完
  成了故障诊断模型的设计。在建立这个多尺度网络结构时,我们在每层递
  归中都将网络分为两个分支,对一个分支中的特征进行下采样、卷积和上
  采样操作,对另一个分支只进行卷积操作,最后将两个分支融合的结果作
  为输出返回上层递归。这样设计的优势在于能够将下采样前包含更多细节
  信息的结果保留下来,并且跟下采样后获得的包含更多整体特征的结果进
  行融合,提升模型对输入信号的理解能力。

  通过仿真实验对比了三种模型在滚动轴承故障数据库CWRU上的诊断效果,
  验证了多尺度特征提取方法在故障诊断算法中的优势。
\end{cabstract}

% 如果习惯关键字跟在摘要文字后面,可以用直接命令来设置,如下:
% \ckeywords{\TeX, \LaTeX, CJK, 模板, 论文}

\begin{eabstract}
   An abstract of a dissertation is a summary and extraction of research work
   and contributions. Included in an abstract should be description of research
   topic and research objective, brief introduction to methodology and research
   process, and summarization of conclusion and contributions of the
   research. An abstract should be characterized by independence and clarity and
   carry identical information with the dissertation. It should be such that the
   general idea and major contributions of the dissertation are conveyed without
   reading the dissertation.

   An abstract should be concise and to the point. It is a misunderstanding to
   make an abstract an outline of the dissertation and words ``the first
   chapter'', ``the second chapter'' and the like should be avoided in the
   abstract.

   Key words are terms used in a dissertation for indexing, reflecting core
   information of the dissertation. An abstract may contain a maximum of 5 key
   words, with semi-colons used in between to separate one another.
\end{eabstract}

% \ekeywords{\TeX, \LaTeX, CJK, template, thesis}
